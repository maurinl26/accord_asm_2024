\documentclass{beamer}
\usepackage{xcolor}
\usepackage{hyperref}
\usepackage{perpage} 	% Numeroter footnote par page
\usepackage{fontawesome5}
\MakePerPage{footnote}
\usetheme{Singapore}
\title[Accord All Staff Meeting]{Towards a Portable Model for All-scale Predictions}

\subtitle{Accord All Staff Meeting}

\author{Loïc Maurin, Fabrice Voitus}

\institute[CNRM]{Centre National de Recherche Météorologique}

\date{April 18th, 2024}
\titlegraphic{
    \includegraphics[height=1cm]{png/logos/logo_cnrm.jpg}
    \includegraphics[height=1cm]{png/logos/logo-mf-blanc-fond-bleu-baseline-jpeg.jpg}
}
\AtBeginSection[]
{
  \begin{frame}{Table of contents}
      \tableofcontents[currentsection]
  \end{frame}
}
\renewcommand{\arraystretch}{1.5}
\addtobeamertemplate{navigation symbols}{}{%
    \usebeamerfont{footline}%
    \usebeamercolor[fg]{footline}%
    \hspace{1em}%
    \insertframenumber/\inserttotalframenumber
}
\setbeamercolor{footline}{fg=blue}
\setbeamerfont{footline}{series=\bfseries}
\setbeamertemplate{caption}[numbered]
%%%%%%%%%%%%%%%%%%%%%%%
\begin{document}
\frame{\titlepage}

\begin{frame}{Table of contents}
    \tableofcontents
\end{frame}

\section{Introduction}

\begin{frame}{Building a reliable dynamical core for high resolutions NWP applications}

    \begin{block}{Improving stability over steep orography}
    \end{block}
    \begin{block}{Maintaining scalability over highly parallel HPC-clusters}
    \end{block}

\end{frame}
    
\begin{frame}{AROME dynamical core}

\end{frame}

\begin{frame}{PMAP - FVM dynamical core}

\end{frame}

\begin{frame}{AROME and FVM dynamical cores - Summary}
    \begin{center}

        \resizebox{\textwidth}{!}{

        \begin{tabular}{|c|c|c|}
            \hline
              & AROME & FVM \\
            \hline
            Vertical coordinate & Mass-based & Height-based  \\
            \hline
            Discretization & Spectral Transform (ST) & Finite Volumes (FV)\\
            \hline
            Linearization & Constant Coefficients & Non-constant Coef. (NC) \\
            \hline
            Implicit solver & Direct & Krylov Methods \\
            \hline 
            Advection & Semi-Lagrangian (SL) & Eulerian (MPDATA)\\
            \hline 
        \end{tabular}
        }
    \end{center}

    \begin{itemize}
        \item Mass-based : hybrid terrain pressure following
        \item Height-based : SLEVE - smooth level vertical coordinate
    \end{itemize}

\end{frame}

\section{From AROME to PMAP-FVM}

\begin{frame}{Transport scheme : Semi-Lagrangian versus MPDATA}

    \begin{block}{Semi-Lagrangian Advection scheme}
    \end{block}
\end{frame}

\begin{frame}{Solver : improving numerical efficiency}
    
    \begin{block}{Direct spectral solver (AROME)}
        \textcolor{blue}{\faIcon{plus}} Exact solver (no iterations) \newline 
        \textcolor{red}{\faIcon{minus}} Heavy global communications
    \end{block}

    \begin{block}{Krylov iterative solvers}
        FVM : Gradient Conjugate Residuals of order k - GCR(k)  \newline
        \textcolor{blue}{\faIcon{plus}} \newline 
        \textcolor{red}{\faIcon{minus}} 
    \end{block}

    \begin{alertblock}{Multigrid preconditioner - \textit{to be investigated}}
    \end{alertblock}
\end{frame}



\begin{frame}{Solver : improving stability over steep orography}
    \begin{block}{Impliciting terms for orography}
        Need for a non-constant Coefficients Helmoltz solver
    \end{block}

    \begin{block}{AROME - Constant Coefficients in hydrostatic coordinate}
        \begin{itemize}
            \item hydrostatic pressure implicitely solved within $\eta$ coordinate
            \item number of steps towards convergence are known in advance
        \end{itemize}

    \end{block}

    \begin{block}{FVM - Non-constant Coefficients}
        
    \end{block}

    \begin{alertblock}{Non-constant coefficient solver in hydrostatic coordinate}
    \end{alertblock}
\end{frame}

\begin{frame}{Summary}

    \begin{block}{Integration of metric terms in Semi-Implicit solver}

    \end{block} 

    \begin{block}{Infusing PMAP with AROME's strengths}
        mass-based coordinate for implicit NH preconditionning 
    \end{block}
    
\end{frame}




%%%%%%%%%%
\section{First Experiments on FVM}

\begin{frame}{No flow over a gaussian orography}

\end{frame}

\begin{frame}{2D vertical flow over a steep orography}

    \begin{center}
        \begin{tabular}{|c|c|c|c|c|}
            \hline
              & $CFL$ & $Lipschitz$ & $\delta t$ & $\delta x$\\
            \hline
            $15^{\circ}$ &  &  &  & \\
            \hline
            $30^{\circ}$ &  &  &  & \\
            \hline
            $45^{\circ}$ &  &  &  & \\
            \hline
            $60^{\circ}$ &  &  &  & \\
            \hline 
            $75^{\circ}$ &  &  &  & \\
            \hline 
        \end{tabular}       
    \end{center}

    \begin{block}{Lipschitz' Deformation constraint} 
        $L = max(\frac{\delta u}{\delta x}, \frac{\delta v}{\delta y}, \frac{\delta w}{\delta z})$
    \end{block}

\end{frame}

\begin{frame}{No flow over the Alps}
\end{frame}

%%%%%%%%%%%
\section{Towards a Demonstrator}

\begin{frame}{Adding Physical processes to PMAP-FVM}

\begin{block}{Refactoring ICE3 Microphysics scheme into GT4Py}
\end{block}

\begin{block}{}
\end{block}


ecRad ported by ETHZ
\end{frame}

\begin{frame}{Building on GT4Py + DaCe}

\begin{block}{GT4Py : GridTools for Python}
    \begin{itemize}
        \item Python Domain System Language (DSL) for HPC code generation
        \item Portable accross CPU and GPU architectures (including AMD backends)
        \item Python code : enhancing readability and Object Oriented Programming
    \end{itemize}    
\end{block}

\begin{block}{DaCe : Data Centric Parallel Programming}
    \begin{itemize}
        \item Optimizing memory allocation for stencils
    \end{itemize}
\end{block}

\begin{block}{DaCeML : Merging AI and Physics based models}
    \begin{itemize}
        \item Model inference using ONNX and integration with Pytorch
        \item Automatic differenciation engine
    \end{itemize}

\end{block}

\end{frame}


%%%%%%%%%%%
\section{Future work}
\begin{frame}{Porting Physics on GT4Py}


\end{frame}

\begin{frame}{Solver}
\end{frame}


\end{document}