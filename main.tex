\documentclass{beamer}
\usepackage{xcolor}
\usepackage{hyperref}
\usepackage{perpage} 	% Numeroter footnote par page
\MakePerPage{footnote}
\usetheme{Singapore}
\title[Accord All Staff Meeting]{Towards a Portable Model for All-scale Predictions}

\subtitle{Accord All Staff Meeting}

\author{Fabrice Voitus, Loïc Maurin}

\institute[CNRM]{Centre National de Recherche Météorologique}

\date{April 18th, 2024}
\titlegraphic{
    \includegraphics[height=1cm]{png/logos/logo_cnrm.jpg}
    \includegraphics[height=1cm]{png/logos/logo-mf-blanc-fond-bleu-baseline-jpeg.jpg}
}
\AtBeginSection[]
{
  \begin{frame}{Sommaire}
      \tableofcontents[currentsection]
  \end{frame}
}
\renewcommand{\arraystretch}{1.5}
\addtobeamertemplate{navigation symbols}{}{%
    \usebeamerfont{footline}%
    \usebeamercolor[fg]{footline}%
    \hspace{1em}%
    \insertframenumber/\inserttotalframenumber
}
\setbeamercolor{footline}{fg=blue}
\setbeamerfont{footline}{series=\bfseries}
\setbeamertemplate{caption}[numbered]
%%%%%%%%%%%%%%%%%%%%%%%
\begin{document}
\frame{\titlepage}
\section{Dynamical Core - AROME and PMAP-FVM}

\begin{frame}{AROME and FVM dynamical cores}
    \begin{center}
        \begin{tabular}{|c|c|c|}
            \hline
             . & AROME & FVM \\
            \hline
            Vertical coordinate & Mass-based & Height-based  \\
            \hline
            Discretization & Spectral Transform (ST) & Finite Volumes (FV)\\
            \hline
            Linearization & Constant Coefficients & Non-constant Coef. (NC) \\
            \hline
            Implicit solver & Direct & Krylov Methods \\
            \hline 
            Advection & Semi-Lagrangian (SL) & Eulerian (MPDATA)\\
            \hline 
        \end{tabular}
    \end{center}

    \begin{itemize}
        \item Mass-based : hybrid terrain pressure following
        \item Z-based : SLEVE - smooth level vertical coordinate
    \end{itemize}

\end{frame}

\begin{frame}{From AROME to FVM}
    Improving stability over steep orography
    \begin{itemize}
        \item Impliciting terms for orography $\rightarrow$ Non-constant Coefficient solver
    \end{itemize}

    AROME   
\end{frame}

\begin{frame}{From AROME to FVM}
    Improving numerical efficiency over new architectures 
    \begin{itemize}
        \item Evolving from Spectral Transform Direct Solver $\rightarrow$ Krylov Iterative Methods
        \item GCR(k) Solver in FVM : good weak scalability compared to direct solver but global communcations at each step
        \item Multigrid : reduced number of global communication thanks to reduction / expansion steps.
    \end{itemize}

\end{frame}

\begin{frame}{Shadow zones of FVM}
    Improving numerical efficiency over new architectures 
    
\end{frame}



\begin{frame}{Semi-Lagrangian Advection versus MPDATA}
\end{frame}




%%%%%%%%%%
\section{First Experiments on FVM}

\begin{frame}{Idealized cases}
\end{frame}

\begin{frame}{No flow over the Alps}
\end{frame}

%%%%%%%%%%%
\section{Towards a Proof of Concept}

\begin{frame}{Coupling PMAP-FVM with physical parametrizations}

Translation of ICE3 Microphysics into GT4Py DSL
\end{frame}

\begin{frame}{Building on GT4Py : A portable python Domain System Language}


\end{frame}


%%%%%%%%%%%
\section{Future work}
\begin{frame}{Porting Physics on GT4Py}


\end{frame}

\begin{frame}{Solver}
\end{frame}


\end{document}